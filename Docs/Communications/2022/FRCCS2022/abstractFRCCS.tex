
\documentclass[12pt]{article}

\usepackage[margin=2cm]{geometry}

\title{\vspace{-2cm}Multi-modeling urban systems dynamics to explore sustainability trade-offs}
\author{J. Raimbault$^{1,\ast}$ and D. Pumain$^2$\\
$^1$ LASTIG, Univ Gustave Eiffel, IGN-ENSG\\
$^2$ UMR CNRS 8504 G{\'e}ographie-cit{\'e}s\\
$\ast$ \texttt{juste.raimbault@polytechnique.edu}
}
\date{}

\begin{document}

\pagenumbering{gobble}

\maketitle

% TODO move to proper doc/repository

% KWs 
% Sustainable Developments Goals
% Urban systems dynamics
% Multi-modeling
% Model optimisation

\noindent

Cities and urban systems appear as key elements to tackle the challenges of sustainability, as they combine negative and positive externalities on multiple dimensions, including the environmental and the socio-economic aspects. Besides, empirical, modeling and theoretical evidence suggest the existence of trade-offs between different Sustainable Development Goals (SDGs) in urban systems. We propose in this contribution to investigate such trade-offs from a modeling perspective. Building on previous work benchmarking macroscopic models of urban systems dynamics (Raimbault, Denis \& Pumain, 2020) based on the evolutionary urban theory (Pumain, 2018), we introduce in a multi-modeling approach the coupling of several layers for complementary dimensions of urban systems. More precisely, we combine an innovation diffusion model, an economic exchange model, and a transportation network model, with a common core of population dynamics. The resulting macroscopic model is parametrised on population data for large urban systems worldwide (Pumain et al., 2015), and integrated into the OpenMOLE platform for model exploration and validation (Reuillon et al., 2013). This allows running a multi-objective optimisation algorithm on proxy indicators for multiple goals: innovation (goal 8 ``Economic Growth''), transportation network (goal 9 ``Infrastructure''), economic inequalities between cities (goal 10 ``Inequalities''), and greenhouse gases emissions (goal 13 ``Climate''). We therein confirm the existence of trade-offs in macroscopic urban dynamics by obtaining Pareto front between these four dimensions. Future perspectives include the extension to multi-scale models to include intra-urban dimensions (and some aspects of goal 11 ``Sustainable cities and societies'') and further model integration to account for other dimensions and goals. Altogether, this horizontal and vertical model integration approach builds the foundations towards multi-scale policies for sustainable territories (Raimbault, 2021).

%\subsection*{References}
\vspace{1cm}

\noindent Pumain, D. (2018). An evolutionary theory of urban systems. In International and transnational perspectives on urban systems (pp. 3-18). Springer, Singapore.

\noindent Pumain, D., Swerts, E., Cottineau, C., Vacchiani-Marcuzzo, C., Ignazzi, C. A., Bretagnolle, A., ... \& Baffi, S. (2015). Multilevel comparison of large urban systems. Cybergeo: European Journal of Geography.

\noindent Raimbault, J. (2021). Integrating and validating urban simulation models. In French Regional Conference on Complex Systems.

\noindent Raimbault, J., Denis, E., \& Pumain, D. (2020). Empowering urban governance through urban science: Multi-scale dynamics of urban systems worldwide. Sustainability, 12(15), 5954.

\noindent Reuillon, R., Leclaire, M., & Rey-Coyrehourcq, S. (2013). OpenMOLE, a workflow engine specifically tailored for the distributed exploration of simulation models. Future Generation Computer Systems, 29(8), 1981-1990.




\end{document}


